\documentclass[xcolor={usenames,dvipsnames}]{beamer}

\usepackage[utf8x]{inputenc}

\definecolor{solarbg}{HTML}{FDF6E3}

\usepackage{minted}
\newminted{c}{bgcolor=solarbg,frame=single}
\usemintedstyle{solarized}

\usetheme{Antibes}
\usecolortheme[named=Bittersweet]{structure}

\title[Glift: Generic, Efficient, Random-Access GPU Data Structures]%
      {Glift:\\ Generic, Efficient, Random-Access GPU Data Structures}
\author{Jean Niklas L'orange}
\institute{\texttt{jeannikl@hypirion.com}}
\date{October 15, 2013}

\begin{document}

\begin{frame}
  \titlepage
\end{frame}

\section{What and why?}
\begin{frame}
  \frametitle{What is this paper about? What is Glift?}

  This paper is about a data-parallel programming abstraction. The abstraction
  enables GPU programmers to write high-level, efficient random-access GPU data
  structures, without sacrificing performance.

  \vfill \pause

  Glift is an implementation of that abstraction, using C++ and Cg.
\end{frame}

\begin{frame}
  \frametitle{Paper Structure}

  This paper is structured into several parts:
  \begin{itemize}
  \item<2-> Rationale, considerations and the abstraction itself
  \only<6->{\item<5-> \textcolor{gray}{(Glift programming)}}
  \item<3-> Classification of existing GPU data structures based on the
    abstraction
  \item<4-> Case studies
  \item<5-> Results, conclusions
    \only<-5>{\item<6->}
  \end{itemize}
\end{frame}

\section{Rationale, considerations, the abstraction}
\subsection{Rationale}
\begin{frame}
  \frametitle{Rationale}

  Why is a data structure abstraction desirable? Two main reasons:
  \begin{enumerate}
  \item<2-> Code reuse -- No need to copypaste existing data structures
  \item<3-> Decouple data structures and algorithms → reduce complexity
  \end{enumerate}
\end{frame}

\end{document}
