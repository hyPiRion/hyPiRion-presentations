\documentclass{beamer}
\usepackage[utf8x]{inputenc}

\usepackage{color}
\usepackage{xcolor}

\definecolor{solarbg}{HTML}{FDF6E3}
\definecolor{Green}{HTML}{00BB00}

\usepackage{minted}
\newminted{java}{bgcolor=solarbg,frame=single}
\newminted{clj}{bgcolor=solarbg,frame=single}
\newminted{py}{bgcolor=solarbg,frame=single}
\usemintedstyle{solarized}

\usetheme{Antibes}
\usecolortheme[named=Green]{structure}

\titlegraphic{\includegraphics[scale=0.75]{img/clj.pdf}}
\title[Introduksjonskurs til programmeringsspråket Clo{\em j}ure]{Kurs i Clo{\em j}ure}
\author{Jean Niklas L'orange}
\date{16. oktober, 2012}

\begin{document}

\begin{frame}
  \titlepage
\end{frame}

\begin{frame}
  \frametitle{Hva er Clo{\em j}ure?}
  \begin{itemize}
  \item<1-> Dynamisk
    \begin{itemize}
      \item<2-> en ny lisp, ikke Scheme eller Common Lisp
    \end{itemize}
  \item<3-> Funksjonelt
    \begin{itemize}
      \item<4-> med fokus på immutabilitet
    \end{itemize}
  \item<5-> Designet for ``flertråding''
  \item<6-> Kjører på Java VM
    \begin{itemize}
      \item<7-> kompilerer ned til Java bytekode
    \end{itemize}
  \item<8-> Ikke objektorientert 
  \end{itemize}
\end{frame}

\begin{frame}
  \frametitle{Motivasjon}
  \framesubtitle{Hvorfor burde man lære seg Clo{\em j}ure?}
  \begin{itemize}
    \item<1-> Lett(ere) å lese og forstå kode
    \item<2-> Designet for programmering med flere tråder
    \item<3-> Har et REPL - lettere å teste/lære språket
    \item<4-> Transaksjonsbasert minne
    \item<5-> Lettere å innføre enn andre språk der du jobber
    \item<6-> Mer morsomt enn andre språk! \pause (\ldots subjektivt!)
  \end{itemize}
\end{frame}

\begin{frame}[fragile]
  \frametitle{Lett(ere) å lese kode}

  Hello world i Java:
  \vspace{3mm}
  \begin{javacode}
public class HelloWorld {
  public static void main(String[] args){
    System.out.println("Hello world!");
  }
}
  \end{javacode}

  versus hello world i Clo{\em j}ure*
  \vspace{3mm}
  \begin{cljcode}
(defn -main [& args]
  (print "Hello world!"))
  \end{cljcode}
\end{frame}

\begin{frame}[fragile]
  \frametitle{Lett(ere) å forstå kode}
Enkel kode i Python:
\vspace{2mm}
  \begin{pycode}
x = [0]
process(x)
print x
  \end{pycode}

Hva skriver den ut? \pause Umulig å vite uten å kjenne til {\tt process}. \pause
\vspace{3mm}
  \begin{cljcode}
(def x [0])
(process x)
(print x)
  \end{cljcode}

Hva skriver denne ut? \pause Skriver alltid ut {\tt [0]}!
\end{frame}

\begin{frame}
  \frametitle{REPL}

  \begin{itemize}
  \item<1-> Kan raskt teste at små deler av koden fungerer
  \item<2-> Er du usikker på om det funker, bare test det ut!
  \end{itemize}
\end{frame}

\begin{frame}
  \frametitle{Transaksjonsbasert minne}

  \begin{itemize}
  \item<1-> Unngår låser, semaforer, deadlocks, etc\ldots
  \item<2-> Kan heller fokusere på problemet programmet skal løse, fremfor at
    programmet deadlocker i ny og ne.
  \end{itemize}
\end{frame}

\end{document}
