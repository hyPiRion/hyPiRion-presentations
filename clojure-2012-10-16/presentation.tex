\documentclass{beamer}
\usepackage[utf8x]{inputenc}

\usepackage{color}
\usepackage{xcolor}

\definecolor{solarbg}{HTML}{FDF6E3}
\definecolor{Green}{HTML}{00BB00}

\usepackage{minted}
\newminted{java}{bgcolor=solarbg,frame=single}
\newminted{clj}{bgcolor=solarbg,frame=single}
\newminted{py}{bgcolor=solarbg,frame=single}
\usemintedstyle{solarized}

\usetheme{Antibes}
\usecolortheme[named=Green]{structure}

\titlegraphic{\includegraphics[scale=0.75]{img/clj.pdf}}
\title[Introduksjonskurs til programmeringsspråket Clo{\em j}ure]{Kurs i Clo{\em j}ure}
\author{Jean Niklas L'orange}
\institute{\texttt{jeannikl@hypirion.com}}
\date{16. oktober, 2012}

\begin{document}

\begin{frame}
  \titlepage
\end{frame}

\section{Hva er Clo{\em j}ure?}

\begin{frame}
  \frametitle{Hva er Clo{\em j}ure?}
  \begin{itemize}
  \item<1-> Dynamisk
    \begin{itemize}
      \item<2-> en ny lisp, ikke Scheme eller Common Lisp
    \end{itemize}
  \item<3-> Funksjonelt
    \begin{itemize}
      \item<4-> med fokus på immutabilitet
    \end{itemize}
  \item<5-> Designet for ``flertråding''
  \item<6-> Kjører på Java VM
    \begin{itemize}
      \item<7-> kompilerer ned til Java bytekode
    \end{itemize}
  \item<8-> Ikke objektorientert 
  \end{itemize}
\end{frame}

\section{Motivasjon}

\begin{frame}
  \frametitle{Motivasjon}
  \framesubtitle{Hvorfor burde man lære seg Clo{\em j}ure?}
  \begin{itemize}
    \item<1-> Lett(ere) å lese og forstå kode
    \item<2-> Designet for programmering med flere tråder
    \item<3-> Har et REPL - lettere å teste/lære språket
    \item<4-> Transaksjonsbasert minne
    \item<5-> Lettere å innføre enn andre språk der du jobber
    \item<6-> Mer morsomt enn andre språk! \pause (\ldots subjektivt!)
  \end{itemize}
\end{frame}

\begin{frame}[fragile]
  \frametitle{Lett(ere) å lese kode}

  Hello world i Java:
  \vspace{3mm}
  \begin{javacode}
public class HelloWorld {
  public static void main(String[] args){
    System.out.println("Hello world!");
  }
}
  \end{javacode}

  versus hello world i Clo{\em j}ure*
  \vspace{3mm}
  \begin{cljcode}
(defn -main [& args]
  (print "Hello world!"))
  \end{cljcode}
\end{frame}

\begin{frame}[fragile]
  \frametitle{Lett(ere) å forstå kode}
Enkel kode i Python:
\vspace{2mm}
  \begin{pycode}
x = [0]
process(x)
print x
  \end{pycode}

\pause Hva skriver den ut? \pause Umulig å vite uten å kjenne til {\tt
  process}. \pause
\vspace{3mm}
  \begin{cljcode}
(def x [0])
(process x)
(print x)
  \end{cljcode}

Hva skriver denne ut? \pause Skriver alltid ut {\tt [0]}!
\end{frame}

\begin{frame}
  \frametitle{REPL}

  \begin{itemize}
  \item<1-> Kan raskt teste at små deler av koden fungerer
  \item<2-> Er du usikker på om det funker, bare test det ut!
  \end{itemize}
\end{frame}

\begin{frame}
  \frametitle{Transaksjonsbasert minne}

  \begin{itemize}
  \item<1-> Unngår låser, semaforer, deadlocks, etc\ldots
  \item<2-> Kan heller fokusere på problemet programmet skal løse, fremfor at
    programmet deadlocker i ny og ne.
  \end{itemize}
\end{frame}

\section{Introduksjon og syntaks}
\subsection{Installasjon og oppsett}
\begin{frame}
  \frametitle{Installasjon}

  Hvordan installerer man Clo{\em j}ure på maskinen sin? Veldig lett og rett
  fram:

  \begin{columns}
    \begin{column}{0.7\textwidth}
       \begin{itemize}
       \item<2-> Leiningen 2.0 er alt du trenger.
       \item<3-> Kan også installere {\tt clojure1.4} på Linux.
       \item<4-> Nødløsning: Last ned Clojure fra clojure.org/downloads og kjør
         \colorbox{solarbg}{\tt java -cp}
         \colorbox{solarbg}{\tt clojure-1.4.0.jar clojure.main}.
  \end{itemize}
    \end{column}
    \begin{column}{0.3\textwidth}
      \uncover<2->{\includegraphics[scale=0.1]{img/leiningen}}
    \end{column}
  \end{columns}
\end{frame}

\begin{frame}
  \frametitle{Oppsett}

  \uncover<1->{Hovedsaklig to store editorer med god støtte for Clo{\em j}ure:}
  \begin{columns}[c]
    \begin{column}{0.4\textwidth}
      \begin{block}<2->{Emacs med clojure-mode}
        \centering
        \includegraphics[height=1cm]{img/emacs}
        \begin{itemize}
        \item<3-> Lett å sette opp
        \item<4-> Lett å mislike oppsettet til emacs
        \end{itemize}
      \end{block}
    \end{column}
    \begin{column}<5->{0.4\textwidth}
      \begin{block}{Eclipse med CCW}
        \centering
        \includegraphics[height=1cm]{img/eclipse}
        \begin{itemize}
        \item<6-> Lett å sette opp
        \item<7-> Ikke designed for lisp, noe emacs er
        \end{itemize}
      \end{block}
    \end{column}
  \end{columns}
  \vspace{3mm}
  \uncover<8->{Om du er glad i Vim, prøv ut {\tt evil}-pakken til Emacs! Da kan
    du bruke pakkene i Emacs, og tasteoppsettet til Vim.}
\end{frame}

\subsection{Syntaks}
\begin{frame}
  \frametitle{Datastrukturer}
  \begin{center}
    {\Huge Syntaks:\\ Atomer og datastrukturer}
  \end{center}
\end{frame}

\begin{frame}
  \frametitle{Atomer}
\begin{semiverbatim}
\uncover<1->{Strenger: \textcolor{blue}{"Hello world! :D"}}\uncover<2->{,
  Characters: \textcolor{blue}{\\J \\Ø \\1}}


\uncover<3->{Tall:} \textcolor{blue}{\uncover<4->{42} \uncover<5->{15.23}
\uncover<6->{1391/1166} \uncover<7->{12345678987654321N}
\uncover<8->{1.0E-1000M}}

\uncover<9->{Boolske variabler: \textcolor{blue}{true false}}\uncover<10->{,
  Null: \textcolor{blue}{nil}}


\uncover<11->{Regexmønstre: \textcolor{blue}{\#"[A-Za-z0-9\_]+"}}


\uncover<12->{Symboler: \textcolor{blue}{foo bar}}\uncover<13->{,
  Nøkkelord (keywords): \textcolor{blue}{:foo :bar}}
\end{semiverbatim}
\end{frame}

\begin{frame}[fragile]
  \frametitle{Datastrukturer}
  Clojure har hovedsaklig fire typer datastrukturer:
\begin{itemize}
\item<2-> Lister - Vokser foran:
\begin{semiverbatim}
\textcolor{blue}{(a\visible<-7>{\alert<7>{,}} b\visible<-7>{\alert<7>{,}} 5.1\visible<-7>{\alert<7>{,}} d) (hello\visible<-7>{\alert<7>{,}} world)}
\end{semiverbatim}
\item<3-> Vektorer - Vokser bak:
\begin{semiverbatim}
\textcolor{blue}{[42\visible<-7>{\alert<7>{,}} 3.14\visible<-7>{\alert<7>{,}} 1.3e20] [:foo\visible<-7>{\alert<7>{,}} :bar]}
\end{semiverbatim}
\item<4-> Maps - Samme som (Hash)Map i Java, dictionary i Python:
\begin{semiverbatim}
\textcolor{blue}{\{:tid "18:15"\visible<-7>{\alert<7>{,}} :sted "F1, NTNU"\} \{:foo 42\visible<-7>{\alert<7>{,}} 20 "acb"\}}
\end{semiverbatim}
\item<5-> Sett - Samme som (Hash)Set i Java, set i Python:
\begin{semiverbatim}
\textcolor{blue}{#\{1\visible<-7>{\alert<7>{,}} 2\visible<-7>{\alert<7>{,}} :foo\} #\{"Eirik"\visible<-7>{\alert<7>{,}} "Fredrik"\visible<-7>{\alert<7>{,}} "Tina"\visible<-7>{\alert<7>{,}} :unknown\}}
\end{semiverbatim}
\end{itemize}
\uncover<6->{Alle datastrukturene kan nøstes, og er immutable!}\uncover<8>{}
\end{frame}

\begin{frame}
  \frametitle{Syntaks}
  Gratulerer! Nå har dere lært all syntaks som trengs for å kunne Clojure.
  \begin{itemize}
    \item<2-> Datastrukturene {\em er} koden og syntaksen i Clojure.
    \item<3-> Funksjonskall, operasjoner, kontrollstruktur ({\tt if}, {\tt defn}
      etc.) og definisjoner er bare lister med navnet på operasjonen først i
      listen.
    \item<4-> Alt er et uttrykk: Det vil si at alt returnerer en verdi.
  \end{itemize}
\end{frame}

\begin{frame}[fragile, t]
  \frametitle{Python vs. Clojure}
  \begin{columns}[T]
    \begin{column}<1->{0.5\textwidth}
      Python
      \begin{overprint} % TODO: Make an alias for this. This is horrible.
        \onslide<2->
        \begin{pycode*}{gobble=10}
          x = 5
        \end{pycode*}
      \end{overprint}
      \begin{overprint}
        \onslide<3->
        \begin{pycode*}{gobble=10}
          x * (a + b + c)
        \end{pycode*}
      \end{overprint}
      \begin{overprint} % TODO: Make an alias for this. This is horrible.
        \onslide<4->
        \begin{pycode*}{gobble=10}
          range(1, 11, 2)
        \end{pycode*}
      \end{overprint}
      \begin{overprint}
        \onslide<5->
        \begin{pycode*}{gobble=10}
          file.close()
        \end{pycode*}
      \end{overprint}
      \begin{overprint}
        \onslide<6->
        \begin{pycode*}{gobble=10}
          def foo(a,b):
              return a + b
        \end{pycode*}
      \end{overprint}
    \end{column}

    \begin{column}<1->{0.5\textwidth}
      Clojure
      \begin{overprint}
        \onslide<2->
        \begin{cljcode*}{gobble=10}
          (def x 5)
        \end{cljcode*}
      \end{overprint}
      \begin{overprint}
        \onslide<3->
        \begin{cljcode*}{gobble=10}
          (* x (+ a b c))
        \end{cljcode*}
      \end{overprint}
      \begin{overprint}
        \onslide<4->
        \begin{cljcode*}{gobble=10}
          (range 1 11 2)
        \end{cljcode*}
      \end{overprint}
      \begin{overprint}
        \onslide<5->
        \begin{cljcode*}{gobble=10}
          (.close file)
        \end{cljcode*}
      \end{overprint}
      \begin{overprint}
        \onslide<6->
        \begin{cljcode*}{gobble=10}
          (defn foo [a b]
            (+ a b))
        \end{cljcode*}
      \end{overprint}
    \end{column}
  \end{columns}
\end{frame}

\begin{frame}[fragile, t]
  \frametitle{Python vs. Clojure}
  \begin{columns}[T]
    \begin{column}<1->{0.5\textwidth}
      Python
      \begin{overprint}
        \onslide<2->
        \begin{pycode*}{gobble=10}
          for x in xs:
              print x
        \end{pycode*}
      \end{overprint}
      \begin{overprint}
        \onslide<3->
        \begin{pycode*}{gobble=10}
          [x for x in xs
                   if x % 2 == 0]
        \end{pycode*}
      \end{overprint}
      \begin{overprint}
        \onslide<4->
        \begin{pycode*}{gobble=10}
          if x == 0:
              return y
          else:
              return z
        \end{pycode*}
      \end{overprint}
    \end{column}

    \begin{column}<1->{0.5\textwidth}
      Clojure
      \begin{overprint}
        \onslide<2->
        \begin{cljcode*}{gobble=10}
          (doseq [x xs]
            (println x))
        \end{cljcode*}
      \end{overprint}
      \begin{overprint}
        \onslide<3->
        \begin{cljcode*}{gobble=10}
          (for [x xs
                :when (even? x)] x)
        \end{cljcode*}
      \end{overprint}
      \begin{overprint}
        \onslide<4->
        \begin{cljcode*}{gobble=10}
          (if (== x 0)
            y
            z)
        \end{cljcode*}
      \end{overprint}
    \end{column}
  \end{columns}
\end{frame}

\begin{frame}[fragile, t]
  \frametitle{Eksempel på evaluering}
  La oss se litt på hvordan Clojure utfører funksjonskall:
  \begin{overprint}
    \onslide<1-5>
    \begin{semiverbatim}
      \alert<2>{(\alert<3>{*} \alert<4>{3} \alert<5>{(+ 2 3)} (/ 3 18))}
    \end{semiverbatim}
    \begin{itemize}
    \item<2-> Vi starter med en parentes, så vi vet at dette er et
      funksjonskall.
    \item<3-> Vi ser at {\tt *} er et symbol, så vi prøver å lese hva den peker
      på. Den peker på multiplikasjonsfunksjonen.
    \item<4-> {\tt 3} er bare et heltall, så Clojure evaluerer og returnerer
      dette.
    \item<5-> Vi starter her med en parentes, så vi ser at vi må gjøre et
      funksjonskall.
    \end{itemize}
    \onslide<6-9>
    \begin{semiverbatim}
      (* 3 (\alert<6>{+} \alert<7>{2} \alert<8>{3}\alert<9>{)} (/ 3 18))
    \end{semiverbatim}
    \begin{itemize}
    \item<6-> {\tt +} er et symbol, og peker på addisjonsfunksjonen.
    \item<7-> {\tt 2} er et heltall, så vi returnerer det. \uncover<8->{Samme
      gjelder for {\tt 3}.}
    \item<9-> Vi har kommet til slutten av listen, så vi kaller {\tt +} med
      argumentene {\tt 2} og {\tt 3}, og får tilbake {\tt 5}.
    \end{itemize}
    \onslide<10-14>
    \begin{semiverbatim}
      (* 3 \alert<10>{5} \alert<11>{(\alert<12>{/} \alert<13>{3 18}\alert<14>{)}})
    \end{semiverbatim}
    \begin{itemize}
    \item<11-> Ny parentes, så nytt funksjonskall.
    \item<12-> {\tt /} er et symbol, og peker på divisjonsfunksjonen.
    \item<13-> {\tt 3} og {\tt 18} er heltall, så vi returnerer dem som vi har
      gjort tidligere
    \item<14-> Vi har kommet til slutten av listen, og kaller {\tt /} med
      argumentene {\tt 3} og {\tt 18}, og får tilbake {\tt 1/6}.
    \end{itemize}
    \onslide<15->
    \begin{semiverbatim}
      \only<15-16>{(* 3 5 \alert<15>{1/6}\alert<16>{)}}
      \only<17->{5/2}
    \end{semiverbatim}
    \begin{itemize}
    \item<16-> Vi har kommet til slutten av listen, så vi kaller {\tt *} med
      argumentene {\tt 3}, {\tt 5} og {\tt 1/6}. Vi får tilbake {\tt 5/2}.
    \item<17-> Vi er ferdige, og returnerer {\tt 5/2}.
    \end{itemize}
  \end{overprint}

  \uncover<18->{NB: Dette gjelder ikke for ``funksjoner'' som f.eks. {\tt if},
    {\tt def(n)} eller andre spesialoperasjoner.}
\end{frame}

\section{Veien videre}

\begin{frame}
  \frametitle{Hvor skal jeg fortsette?}
  Hvor fortsetter man for å bli bedre i Clojure? Noen forslag:
  \begin{itemize}
  \item<+-> 4Clojure (\url{http://www.4clojure.com}) er en unik kilde for å
    lære mer om Clojure gjennom lette oppgaver.
  \item<+-> ``Juksearket'' (\url{http://clojure.org/cheatsheet}) er en veldig
    praktisk kilde for å lære nye funksjoner.
  \item<+-> Makroene {\tt doc} og {\tt source} i Clojure gir deg dokumentasjon
    og kildekode på alle funksjoner du har tilgjengelig, også fra biblioteker!
  \item<+-> For de som er interessert i matte, så er Project Euler
    (\url{http://projecteuler.net}) en god måte å lære Clojure på.
  \end{itemize}
\end{frame}

\begin{frame}
  \frametitle{Bøker}
  Anbefalte bøker:

\begin{columns}[c]
    \begin{column}{0.4\textwidth}
      \begin{block}<2->{Clojure Programming}
        \centering
        \includegraphics[width=3cm]{img/clj-prog}
      \end{block}
    \end{column}
    \begin{column}<3->{0.4\textwidth}
      \begin{block}{The Joy of Clojure}
        \centering
        \includegraphics[width=3cm]{img/joy-of-clj}
      \end{block}
    \end{column}
  \end{columns}
\vspace{3mm}
\uncover<4->{Begge finnes på Amazon.}
\end{frame}

\begin{frame}
  \frametitle{Miljøet}
  Clojure-communitiet er veldig åpent og hyggelig

  \begin{itemize}
  \item<2-> IRC: {\tt \#clojure} på Freenode
  \item<3-> Epostlister: {\tt Clojure} og {\tt Clojure Dev} på Google Groups
  \item<4-> Biblioteker og rammeverk er det bare å bidra på
  \item<5-> Clojure Core - må sende inn Contributors Agreement
  \end{itemize}
\end{frame}


\end{document}
